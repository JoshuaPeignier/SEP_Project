\documentclass{article}
\usepackage[utf8]{inputenc}
\usepackage[T1]{fontenc}
\usepackage[francais]{babel}
\usepackage[left=2cm,right=2cm,top=2cm,bottom=2cm]{geometry}

\usepackage{amsmath}
\usepackage{tikz}

\title{SEP Project}
\author{Joshua Peignier \and Solène Mirliaz}
\date{\today}

\begin{document}
\maketitle

\section{Protocole}
\begin{align*}
    A & \rightarrow S : senc(<K, hash(B)>, k_{SA})
      & \text{A confirme et donne son message}\\
    S & \rightarrow B : senc(<K, A>, k_{SB})
      & \text{S transmet à B, avec l'identité de A}\\
    B & \rightarrow A : hash(K)
      & \text{B confirme réception (et son identité)}\\
\end{align*}

\paragraph{Connaissances initiales: } Au début du protocole, chaque role A et B connait une clé symmetrique publique $k_{AS}$ et $k_{BS}$. Ces clés sont également connues du role S.

\paragraph{Valeurs générées au cours du protocole: } K est un nonce généré par A au début du protocole.

\paragraph{Description du protocole: } La première étape du protocole est initiée par le A, qui envoie le nonce K et un hachage du destinataire B, le tout encodé avec la clé symmetrique $k_{SB}$ dont il dispose, à S.
%
À la réception, S peut déchiffrer le contenu du message c'est-à-dire K et $hash(K)$. Il peut comparer le hachage reçu avec celui de B pour attester que la communication se fait bien avec B.
%
Ensuite, S transmet le message, avec le nom de l'envoyeur, à B, le tout encodé avec la clé symmetrique $k_{SB}$.
%
B peut décrypter ce message et à donc accès à la donnée K ainsi qu'au nom de l'agent avec lequel il a communiqué.
%
C'est par un simple hachage que la preuve de réception va être transmise de B à A.

\paragraph{Propriétés de sécurité: }
\emph{Authentification}: Le message <K, A> reçu par B, provient bien d'un envoi par A de K à B.
%Un attaquant ne peut pas forger un tel message car il ne dispose pas de la clé $k_{SB}$. Il doit donc le récupérer directement dans un échange entre agents honnêtes. Seul le rôle S peut le forger. Mais S doit au préalable recevoir $senc(<K, hash(B)>, k_{SA})$. Un attaquant ne peut pas forger ce message, car il ne dispose pas de la clé $k_{SA}$. Il doit donc le récupérer. Or, ce message ne peut être forgé que par A, quand il désire communiquer K à B. Le K reçu par B et provenant d'un prétendu agent A ne peut donc être reçu par B que si A l'a effectivement envoyé.
\emph{Confidentialité}: La donnée K reste secrète entre les agents A, B et S. Un attaquant ne peut pas décoder un message pour en apprendre plus sur K.
\emph{Synchronisation}: Si A termine en ayant envoyé K à B, alors B a bien reçu K.

\paragraph{Poids du protocole: }
\begin{itemize}
    \item Règle 1: $1 + ((50 + 1 + 1 + 1) + 1) = 55$
    \item Règle 2: $1 + ((50 + 1 + 1) + 1) = 54$
    \item Règle 3: $1 + 1 = 2$
\end{itemize}
Le coût total est donc de \textbf{111}.
\end{document}

